%  LaTeX support: latex@mdpi.com 
%  For support, please attach all files needed for compiling as well as the log file, and specify your operating system, LaTeX version, and LaTeX editor.

% see also Google doc at https://docs.google.com/document/d/1Hitne_x4s-Iivetij_XnJBoShJj95FDxYIDc6PRfBDQ/edit#heading=h.2usqgv7yajxv
%=================================================================
\documentclass[remotesensing,article,submit,pdftex,moreauthors]{Definitions/mdpi} 
% For posting an early version of this manuscript as a preprint, you may use "preprints" as the journal and change "submit" to "accept". The document class line would be, e.g., \documentclass[preprints,article,accept,moreauthors,pdftex]{mdpi}. This is especially recommended for submission to arXiv, where line numbers should be removed before posting. For preprints.org, the editorial staff will make this change immediately prior to posting.

%--------------------
% Class Options:
%--------------------
%----------
% journal
%----------
% Choose between the following MDPI journals:
% acoustics, actuators, addictions, admsci, adolescents, aerospace, agriculture, agriengineering, agronomy, ai, algorithms, allergies, alloys, analytica, animals, antibiotics, antibodies, antioxidants, applbiosci, appliedchem, appliedmath, applmech, applmicrobiol, applnano, applsci, aquacj, architecture, arts, asc, asi, astronomy, atmosphere, atoms, audiolres, automation, axioms, bacteria, batteries, bdcc, behavsci, beverages, biochem, bioengineering, biologics, biology, biomass, biomechanics, biomed, biomedicines, biomedinformatics, biomimetics, biomolecules, biophysica, biosensors, biotech, birds, bloods, blsf, brainsci, breath, buildings, businesses, cancers, carbon, cardiogenetics, catalysts, cells, ceramics, challenges, chemengineering, chemistry, chemosensors, chemproc, children, chips, cimb, civileng, cleantechnol, climate, clinpract, clockssleep, cmd, coasts, coatings, colloids, colorants, commodities, compounds, computation, computers, condensedmatter, conservation, constrmater, cosmetics, covid, crops, cryptography, crystals, csmf, ctn, curroncol, currophthalmol, cyber, dairy, data, dentistry, dermato, dermatopathology, designs, diabetology, diagnostics, dietetics, digital, disabilities, diseases, diversity, dna, drones, dynamics, earth, ebj, ecologies, econometrics, economies, education, ejihpe, electricity, electrochem, electronicmat, electronics, encyclopedia, endocrines, energies, eng, engproc, ent, entomology, entropy, environments, environsciproc, epidemiologia, epigenomes, est, fermentation, fibers, fintech, fire, fishes, fluids, foods, forecasting, forensicsci, forests, foundations, fractalfract, fuels, futureinternet, futureparasites, futurepharmacol, futurephys, futuretransp, galaxies, games, gases, gastroent, gastrointestdisord, gels, genealogy, genes, geographies, geohazards, geomatics, geosciences, geotechnics, geriatrics, hazardousmatters, healthcare, hearts, hemato, heritage, highthroughput, histories, horticulturae, humanities, humans, hydrobiology, hydrogen, hydrology, hygiene, idr, ijerph, ijfs, ijgi, ijms, ijns, ijtm, ijtpp, immuno, informatics, information, infrastructures, inorganics, insects, instruments, inventions, iot, j, jal, jcdd, jcm, jcp, jcs, jdb, jeta, jfb, jfmk, jimaging, jintelligence, jlpea, jmmp, jmp, jmse, jne, jnt, jof, joitmc, jor, journalmedia, jox, jpm, jrfm, jsan, jtaer, jzbg, kidney, kidneydial, knowledge, land, languages, laws, life, liquids, literature, livers, logics, logistics, lubricants, lymphatics, machines, macromol, magnetism, magnetochemistry, make, marinedrugs, materials, materproc, mathematics, mca, measurements, medicina, medicines, medsci, membranes, merits, metabolites, metals, meteorology, methane, metrology, micro, microarrays, microbiolres, micromachines, microorganisms, microplastics, minerals, mining, modelling, molbank, molecules, mps, msf, mti, muscles, nanoenergyadv, nanomanufacturing, nanomaterials, ncrna, network, neuroglia, neurolint, neurosci, nitrogen, notspecified, nri, nursrep, nutraceuticals, nutrients, obesities, oceans, ohbm, onco, oncopathology, optics, oral, organics, organoids, osteology, oxygen, parasites, parasitologia, particles, pathogens, pathophysiology, pediatrrep, pharmaceuticals, pharmaceutics, pharmacoepidemiology, pharmacy, philosophies, photochem, photonics, phycology, physchem, physics, physiologia, plants, plasma, pollutants, polymers, polysaccharides, poultry, powders, preprints, proceedings, processes, prosthesis, proteomes, psf, psych, psychiatryint, psychoactives, publications, quantumrep, quaternary, qubs, radiation, reactions, recycling, regeneration, religions, remotesensing, reports, reprodmed, resources, rheumato, risks, robotics, ruminants, safety, sci, scipharm, seeds, sensors, separations, sexes, signals, sinusitis, skins, smartcities, sna, societies, socsci, software, soilsystems, solar, solids, sports, standards, stats, stresses, surfaces, surgeries, suschem, sustainability, symmetry, synbio, systems, taxonomy, technologies, telecom, test, textiles, thalassrep, thermo, tomography, tourismhosp, toxics, toxins, transplantology, transportation, traumacare, traumas, tropicalmed, universe, urbansci, uro, vaccines, vehicles, venereology, vetsci, vibration, viruses, vision, waste, water, wem, wevj, wind, women, world, youth, zoonoticdis 
\usepackage{pythonhighlight}
\usepackage{tcolorbox}

% Addding a package for commenting inline
% \usepackage[margins]{trackchanges}
%---------
% article
%---------
% The default type of manuscript is "article", but can be replaced by: 
% abstract, addendum, article, book, bookreview, briefreport, casereport, comment, commentary, communication, conferenceproceedings, correction, conferencereport, entry, expressionofconcern, extendedabstract, datadescriptor, editorial, essay, erratum, hypothesis, interestingimage, obituary, opinion, projectreport, reply, retraction, review, perspective, protocol, shortnote, studyprotocol, systematicreview, supfile, technicalnote, viewpoint, guidelines, registeredreport, tutorial
% supfile = supplementary materials

%----------
% submit
%----------
% The class option "submit" will be changed to "accept" by the Editorial Office when the paper is accepted. This will only make changes to the frontpage (e.g., the logo of the journal will get visible), the headings, and the copyright information. Also, line numbering will be removed. Journal info and pagination for accepted papers will also be assigned by the Editorial Office.

%------------------
% moreauthors
%------------------
% If there is only one author the class option oneauthor should be used. Otherwise use the class option moreauthors.

%---------
% pdftex
%---------
% The option pdftex is for use with pdfLaTeX. If eps figures are used, remove the option pdftex and use LaTeX and dvi2pdf.

%=================================================================
% MDPI internal commands
\firstpage{1} 
\makeatletter 
\setcounter{page}{\@firstpage} 
\makeatother
\pubvolume{1}
\issuenum{1}
\articlenumber{0}
\pubyear{2022}
\copyrightyear{2022}
%\externaleditor{Academic Editor: Firstname Lastname}
\datereceived{} 
%\daterevised{} % Only for the journal Acoustics
\dateaccepted{} 
\datepublished{} 
%\datecorrected{} % Corrected papers include a "Corrected: XXX" date in the original paper.
%\dateretracted{} % Corrected papers include a "Retracted: XXX" date in the original paper.
\hreflink{https://doi.org/} % If needed use \linebreak
%\doinum{}
%------------------------------------------------------------------
% The following line should be uncommented if the LaTeX file is uploaded to arXiv.org
%\pdfoutput=1

%=================================================================
% Add packages and commands here. The following packages are loaded in our class file: fontenc, inputenc, calc, indentfirst, fancyhdr, graphicx, epstopdf, lastpage, ifthen, lineno, float, amsmath, setspace, enumitem, mathpazo, booktabs, titlesec, etoolbox, tabto, xcolor, soul, multirow, microtype, tikz, totcount, changepage, attrib, upgreek, cleveref, amsthm, hyphenat, natbib, hyperref, footmisc, url, geometry, newfloat, caption

%=================================================================
%% Please use the following mathematics environments: Theorem, Lemma, Corollary, Proposition, Characterization, Property, Problem, Example, ExamplesandDefinitions, Hypothesis, Remark, Definition, Notation, Assumption
%% For proofs, please use the proof environment (the amsthm package is loaded by the MDPI class).

%=================================================================
% Full title of the paper (Capitalized)
\Title{An Ontology for Describing Wind Lidar Concepts}

% MDPI internal command: Title for citation in the left column
\TitleCitation{An Ontology for Describing Wind Lidar Concepts}

% Author Orchid ID: enter ID or remove command
\newcommand{\orcidauthorA}{0000-0003-1318-9677} % Add \orcidA{} behind the author's name
%\newcommand{\orcidauthorB}{0000-0000-0000-000X} % Add \orcidB{} behind the author's name
\newcommand{\orcidauthorC}{0000-0001-7221-9800} % Add \orcidC{} behind the author's name
\newcommand{\orcidauthorG}{0000-0002-8397-7348}
\newcommand{\orcidauthorL}{0000-0001-9698-5083}

% Authors, for the paper (add full first names)
\Author{
Francisco Costa $^{1}$\orcidA{}*,
Ashim Giyanani  $^{2}$\orcidG{},
Dexing Liu $^{3}$,
Aidan Keane $^{4}$\orcidC{},
Carlo Alberto Ratti $^{5}$
and Andrew~Clifton$^{6}$\orcidL{}}

%\longauthorlist{yes}

% MDPI internal command: Authors, for metadata in PDF
\AuthorNames{Francisco Costa, Dexing Liu, Aidan Keane, Ashim Giyanani, Carlo Alberto Ratti and Andrew Clifton}

% MDPI internal command: Authors, for citation in the left column
\AuthorCitation{Costa, F.; Giyanani, A;  Liu, D.; Keane, A; Ratti, C; Clifton, A.}
% If this is a Chicago style journal: Lastname, Firstname, Firstname Lastname, and Firstname Lastname.

% Affiliations / Addresses (Add [1] after \address if there is only one affiliation.)
\address{%
$^{1}$ \quad University of Stuttgart, Allmandring 5b, 70569 Stuttgart ; costa@ifb.uni-stuttgart.de\\

$^{2}$ \quad Fraunhofer IWES, 27572 Bremerhaven, Germany; ashim.giyanani@iwes.fraunhofer.de\\

$^{3}$ \quad University of Stuttgart, Allmandring 5b, 70569 Stuttgart ; liu@ifb.uni-stuttgart.de\\
$^{4}$ \quad Wood Renewables, Floor 2 St Vincent Plaza, 319 St Vincent Street, Glasgow, G2 5LP; aidan.keane@woodplc.com\\
$^{5}$ \quad Enlight Renewable Energy; carloar@enlightenergy.eu\\
$^{6}$ \quad TGU enviConnect, TTI GmbH, Nobelstrasse 15, 70569 Stuttgart, Germany; andy.clifton@enviconnect.de}

% Contact information of the corresponding author
\corres{Correspondence: costa@ifb.uni-stuttgart.de; Tel.: +34 691 020 173}

% Current address and/or shared authorship
% \firstnote{Current address: Affiliation 1.} 
%\secondnote{These authors contributed equally to this work.}
% The commands \thirdnote{} till \eighthnote{} are available for further notes

%\simplesumm{} % Simple summary

%\conference{} % An extended version of a conference paper

% Abstract (Do not insert blank lines, i.e. \\) 
\abstract{
This article reports on an open-source ontology which has been developed to establish an industry-wide consensus on wind lidar concepts and terminology.
The article provides an introduction to the Wind Lidar Ontology and gives an overview of its development, and a summary of the aims and achievements.
The ontology serves both reference and educational purposes for wind energy applications and lidar technology.
The article provides an overview of the creation process, the outcomes of the project and the proposed uses of the ontology.
The ontology is available online and provides standardisation of terminology within the lidar knowledge domain. The open-source framework provides the basis for information sharing and integration within remote sensing science and the fields of application.
}

% Keywords
\keyword{Wind Energy; Wind Lidar; Remote Sensing; Wind Characterisation; Domain Knowledge; Ontology; Open-Source} 

% The fields PACS, MSC, and JEL may be left empty or commented out if not applicable
%\PACS{J0101}
%\MSC{}
%\JEL{}

%%%%%%%%%%%%%%%%%%%%%%%%%%%%%%%%%%%%%%%%%%
% Only for the journal Diversity
%\LSID{\url{http://}}

%%%%%%%%%%%%%%%%%%%%%%%%%%%%%%%%%%%%%%%%%%
% Only for the journal Applied Sciences
%\featuredapplication{Authors are encouraged to provide a concise description of the specific application or a potential application of the work. This section is not mandatory.}
%%%%%%%%%%%%%%%%%%%%%%%%%%%%%%%%%%%%%%%%%%

%%%%%%%%%%%%%%%%%%%%%%%%%%%%%%%%%%%%%%%%%%
% Only for the journal Data
%\dataset{DOI number or link to the deposited data set if the data set is published separately. If the data set shall be published as a supplement to this paper, this field will be filled by the journal editors. In this case, please submit the data set as a supplement.}
%\datasetlicense{License under which the data set is made available (CC0, CC-BY, CC-BY-SA, CC-BY-NC, etc.)}

%%%%%%%%%%%%%%%%%%%%%%%%%%%%%%%%%%%%%%%%%%
% Only for the journal Toxins
%\keycontribution{The breakthroughs or highlights of the manuscript. Authors can write one or two sentences to describe the most important part of the paper.}

%%%%%%%%%%%%%%%%%%%%%%%%%%%%%%%%%%%%%%%%%%
% Only for the journal Encyclopedia
%\encyclopediadef{For entry manuscripts only: please provide a brief overview of the entry title instead of an abstract.}

%%%%%%%%%%%%%%%%%%%%%%%%%%%%%%%%%%%%%%%%%%
% Only for the journal Advances in Respiratory Medicine
%\addhighlights{yes}
%\renewcommand{\addhighlights}{%

%\noindent This is an obligatory section in “Advances in Respiratory Medicine”, whose goal is to increase the discoverability and readability of the article via search engines and other scholars. Highlights should not be a copy of the abstract, but a simple text allowing the reader to quickly and simplified find out what the article is about and what can be cited from it. Each of these parts should be devoted up to 2~bullet points.\vspace{3pt}\\
%\textbf{What are the main findings?}
% \begin{itemize}[labelsep=2.5mm,topsep=-3pt]
% \item First bullet.
% \item Second bullet.
% \end{itemize}\vspace{3pt}
%\textbf{What is the implication of the main finding?}
% \begin{itemize}[labelsep=2.5mm,topsep=-3pt]
% \item First bullet.
% \item Second bullet.
% \end{itemize}
%}

%%%%%%%%%%%%%%%%%%%%%%%%%%%%%%%%%%%%%%%%%%
\begin{document}

%%%%%%%%%%%%%%%%%%%%%%%%%%%%%%%%%%%%%%%%%%
%\setcounter{section}{-1} %% Remove this when starting to work on the template.
%\section{How to Use this Template}

%The template details the sections that can be used in a manuscript. Note that the order and names of article sections may differ from the requirements of the journal (e.g., the positioning of the Materials and Methods section). Please check the instructions on the authors' page of the journal to verify the correct order and names. For any questions, please contact the editorial office of the journal or support@mdpi.com. For LaTeX-related questions please contact latex@mdpi.com.%\endnote{This is an endnote.} % To use endnotes, please un-comment \printendnotes below (before References). Only journal Laws uses \footnote.

% The order of the section titles is different for some journals. Please refer to the "Instructions for Authors” on the journal homepage.

\section{Introduction}
Many aspects of wind resource assessment and inflow characterization require accurate measurement of wind speed, wind direction and turbulence. Wind measurement data can be provided by meteorological mast instrumentation or remote sensing devices. In recent years, measurement of wind speeds with lasers, based upon the lidar (Light Detection And Ranging) system, has provided a modern alternative to the traditional meteorological mast instrumentation. Wind lidar makes use of the principle of optical Doppler shift between the reference radiation and radiation backscattered by aerosol particles to measure radial wind velocities at distances up to several kilometers \cite{Pena_Hasager_2010,ref-Liu, vanDooren2020,Shangguan_2022}.
In fields such as aviation, climatology, satellite-based measurements and especially wind energy, there has been a general move towards adoption of lidar as a cost effective, self-contained and adaptable solution to wind measurement.
The use of wind lidars is now widely accepted within the wind energy sector and there has been an associated increase in the technological development of lidar devices and in use cases, and incorporation within a number of IEC standards \cite{ref-IEC61400-12-1, ref-IEC61400-50-2, ref-IEC61400-50-3}.

The development of new sensors and the accessibility of data sets are significantly transforming both the theory and practice of remote sensing.
The interest in wind lidar has led to it becoming a research field in its own right, and as a result there has been a multitude of technical advancements and a corresponding plethora of lidar specific concepts and terminology. 
Further, lidar manufacturers use a variety of different concept definitions, conventions and data formats.
As a result, there exist many difficulties and challenges in communication and lidar knowledge transfer, which may be the case in particular for early stage researchers and engineers. This situation has given rise to the need for a system to ensure consistency and comprehension within the field.

To address these issues, academia and industry have been working towards consensus and standardisation, and several examples are notable:
The global initiative project e-Wind Lidar \cite{ref-e-Wind-Lidar} focused on development of wind lidar community based tools and data standards, and published a lidar data form to make them Findable, Accessible, Interoperable and Reusable (FAIR) \cite{ref-FAIR}; The IEA Wind TCP Task 52 \cite{ref-IEA-Wind-Task-52} investigates the challenges to the large-scale adoption of wind lidar and is the successor to IEA Wind TCP Task 32, which set out to identify and mitigate the barriers to the adoption of wind lidar \cite{ref-Clifton-Schlipf}. The OpenLidar project \cite{clifton_andrew_2019_3414197} -- coordinated by the Stuttgart Chair of Wind Energy -- developed an architecture of an open-source remote wind sensing lidar that can be used for teaching, research, and product development.
\citet{Marykovskiy2023} recognised the need to create value from the data available within the wind energy domain and addressed the challenges faced by wind energy domain experts in converting data into domain knowledge, and integration with other sources of knowledge. Their work presents an extensive overview of existing wind energy domain semantic models (i.e. ontologies, taxonomies, schema, etc.) and highlights the role of knowledge engineering within the digital transformation of the wind energy sector.

While these initiatives have been successful, further effort is required to capitalise on their achievements and bring further consensus and standardisation within a growing lidar community.
In recognition of this, an open-source ontology has been developed with the aim of establishing an industry-wide consensus on wind lidar concepts and terminology and their relationships.
An ontology defines a common vocabulary within a domain, providing a means to share information in the domain, and includes machine-interpretable definitions of fundamental concepts and the relations among them.
Many disciplines now develop standardised ontologies that domain experts can use to share and annotate information in their fields \cite{ref-Noy}.
An ontology serves several purposes: sharing the common understanding of the structure of information; making the domain assumptions explicit; and, enabling reuse and analysis of the domain knowledge.

The Wind Lidar Ontology \cite{ref-Clifton-Costa} was created as a sub-task of the IEA Wind TCP Task 32, and has been taken on, as already mentioned, by a multi-disciplinary group of academics and industry engineers.
This has resulted in the online publication of the latest version of the Wind Lidar Ontology \cite{ref-IEA-Wind-Task-32-wind-lidar-ontology}.
The Wind Lidar Ontology provides an introduction to the fundamental concepts and terminology relating to lidar hardware and software, lidar installation and measurement, data generation, processing and analysis.
The ontology provides guidelines for a common conceptual architecture for lidar design, modelling and analysis.
The ontology includes the lidar module design according to the OpenLidar Architecture \cite{clifton_andrew_2019_3414197}, atmospheric parameters, data configuration and measurement principles.
The development of knowledge-driven approaches is considered to be one of the most important directions of research by the remote sensing community \cite{ref-Arvor_2019}.
It is hoped that the ontology will play a part in the development of an industry-wide consensus on lidar concepts, and the standardisation of lidar terminology.
The ontology is well placed to facilitate reuse of the domain knowledge and collaboration amongst researchers and engineers.

The Wind Lidar Ontology scope includes wind energy applications and lidar technology.
The Wind Lidar Ontology has been developed as an open-source code repository, with the purpose of allowing
free access to users, with options to contribute and redistribute content.
The Wind Lidar Ontology serves several purposes: a reference source and dictionary for users, and will be of particular
use for new entrants to the field of wind energy;
a variable definition reference for datasets for original equipment manufacturers (OEMs) delivering data to clients;
an input resource for wind modelling and lidar simulations.
As an open-source code repository, the Wind Lidar Ontology will be subject to ongoing development, with individual releases identified with release version numbers.
The definitions are available to be downloaded from the code repository.

It is acknowledged that a number of related ontologies have been proposed within the wider wind science community.
For instance, previous works on digital twins and digitalisation has provided a platform to develop common ontologies and taxonomies within fields related to wind energy. Using the Digital Twins Definition Language (DTDL) and Common Information Model (CIM), an energy grid ontology was developed by Microsoft in cooperation with other industrial partners \cite{Ravi2021}.
However, this ontology is very generic in nature and is not directly applicable to remote sensing using lidar for wind energy purposes. \citet{Kucuk2018} developed OntoWind, a high-level wind energy ontology aimed at providing a consensual definition of wind energy terms, but does not provide details of remote sensing terms.
The data format developed as a part of the European Aerosol Research Lidar Network (EARLINET) \cite{ref-EARLINET}
is based on a common vocabulary agreed by the participating network, however is limited to atmospheric aerosols. The ESIP Lidar Cluster \cite{ref-ESIP} hosts ontologies and vocabularies for lidar-related terms in Earth science research and is open for public to contribute, however most of the ontologies focus on the numerical modelling vocabularies or on electronic aspects related to wind energy measurement. 

It is intended that the development of the Wind Lidar Ontology addresses the aforementioned perceived issues and challenges relating to lidar knowledge within the wind lidar field.
This article provides an introduction to the Wind Lidar Ontology, and gives an overview of its structure and development. Section \ref{sec:Methodology} provides an overview of the ontology methodology. The results are presented in Section \ref{sec:Results} and are discussed in Section \ref{sec:Discussion}. The general outlook for future applications is presented in Section \ref{sec:Outlook}.

\section{Methodology}
\label{sec:Methodology}
The processes and structures underlying the construction and development of the Wind Lidar Ontology are outlined in this section. 

In view of the need to establish an industry-wide consensus on wind lidar concepts and terminology
and their relationships, the IEA Wind TCP Task 52 and,
particularly the Wind Lidar Ontology Working Group \cite{ref-IEA-Wind-Task-52-OntologyGroup}
brought together a variety of researchers and industry stakeholders to collaborate on the
standardisation of lidar knowledge. The core team was assembled with experts possessing
diverse backgrounds within related domains, yet unified in purpose, to foster a synergistic collaboration.
This assemblage comprises representatives from the Fair Data Collective initiative, industrial entities,
and academic institutions including the Stuttgart Wind Energie (SWE) institute at the University
of Stuttgart and the Fraunhofer Institute for Wind Energy Systems (IWES). Tasked in accordance with
their respective proficiencies, these stakeholders collectively contributed to the development of
a comprehensive framework, the conceptualization of the ontology, and the delineation of lidar concepts.
This multi-disciplinary approach allowed for enrichment from diverse perspectives,
collaboratively addressing the varied needs of industry and academia encountered during the project. 

The Wind Lidar Ontology Working Group carried out the following steps to develop
the Wind Lidar Ontology:
\begin{itemize}
    \item Determine the optimal data model format for saving vocabularies and ensuring accessibility.
    \item Populate the concepts and definitions during collaborative discussions.
    \item Translate the concept definitions into a number of languages.
    \item Provide open-source tools to facilitate dissemination within the wider community.
\end{itemize}


\subsection{Data model format}
Well-defined data models were employed in consensus to document and maintain the ontology.
These data models should ideally be a mixture of human- and machine-readable data models that can be presented to users via different interfaces, as required. In the current context, this is achieved using the Resource Description Framework (RDF) \cite{ref-W3C-RDF} and the Simple Knowledge Organisation System (SKOS) \cite{ref-W3C-SKOS}.

A {\it controlled vocabulary} is a carefully selected list of terms within a specific domain of knowledge, which are used to tag units of information so that they may be retrieved by a search.
Metadata may be specified in accordance with a controlled vocabulary, providing clear and concise representation of data sets, and allowing data to be structured, stored, maintained and shared in a consistent manner. Each item within the controlled vocabulary may be regarded as a concept within the specific domain of knowledge.

RDF is a standard model for data interchange on the Web. It is commonly used to model a controlled vocabulary and define properties, relationships, constraints and axioms among its concepts. It therefore offers the opportunity to digitalise, express and handle knowledge organisation systems in a machine-readable format. Similarly, SKOS offers standards to create data models for metadata, and jointly with RDF facilitates the provision of standardised data organisation models to server machines.

A {\it taxonomy} is a type of controlled vocabulary in which concepts are related in a hierarchical order or sorted into categories. An ontology is a type of controlled vocabulary which identifies and distinguishes concepts and their relationships.

The first step in building a {\it lidar taxonomy} is to create a list of terms, or concepts, related to the wind lidar knowledge domain. To do so, the authors followed the "expert elicitation" approach to establish a consensus position. In this approach, experts in the relevant topics gathered in a collaborative way to define the hierarchy of terms and concepts \cite{ref-IRPWind}. Figure \ref{fig:tax} illustrates the highest levels of the lidar taxonomy, from the broadest term "wind lidar" to the narrower terms.

\begin{figure}[ht]
    \centering
    \includegraphics[width=\textwidth]{Figures/MindMap.PNG}
    \caption{The highest levels of the lidar taxonomy, from the broadest term "wind lidar" to narrower terms. First and second hierarchical levels are shown, along with one third level (under "Design") which has been expanded.}
    \label{fig:tax}
\end{figure}

\subsection{Lidar concepts}
When the lidar concepts are endowed with extended information and their interrelationships specified, the lidar taxonomy becomes a {\it Wind Lidar Ontology}. The Wind Lidar Ontology Working Group, encompassing multi-disciplinary fields of knowledge,
contributed to the definition of the concepts, the vocabularies, the metadata and the relationships, through intensive
discussions organised by IEA Wind TCP Task 52.

\subsection{Translation}
The participants in the working group encompass a wide range of cultural and regional backgrounds,
facilitating direct translation between multiple European languages and Chinese language by native speakers.

\subsection{Tools and Implementation}
\label{subsec:implementation}
The wrappers sheet2rdf \cite{ref-Fiorelli2015, nikola_vasiljevic_2021_4432136} and OntoStack \cite{ref-OntoStack} have been used to orchestrate the framework necessary to deploy the Wind Lidar Ontology. The ontology is deployed by means of a combination of several tools, brought together using GitHub Actions \cite{ref-GitActions}, which enable automatic compilation and deployment. The workflow is as follows: 
\begin{description}
    
    \item [Input data sheet:] The input data sheet consists of a Google Sheets spreadsheet stored in Google Drive. The Google sheet is used as the graphical user interface (GUI) to edit the ontology as the GUI simplifies collaboration compared to working on the machine-readable SKOS or RDF formats. The spreadsheet has been populated with the lidar concepts and the related descriptive metadata considered by the authors to be most relevant for the wind lidar community. The source files for the Wind Lidar Ontology and the spreadsheet are hosted in the repository  \textit{IEA-Wind-Task-32/wind-lidar-ontology} \cite{TCP32_GitHub_repo}.
    The Google spreadsheet format is read and manipulated by sheet2rdf (see below) and converted to RDF, facilitating use as an online searchable asset. 
    
    \item [sheet2rdf:] The sheet2rdf \cite{ref-Fiorelli2015, nikola_vasiljevic_2021_4432136} framework builds on the excel2rdf \cite{ref-excel2rdf} workflow, enabling the acquisition and transformation of data sheets into RDF \cite{ref-github-sheet2rdf}. The specific sheet2rdf instance used in the present work is available in a GitHub repository created by the FAIR Data Collective \cite{ref-FAIRsheet2rdf}. It contains the main functions that enable the conversion of the ontology information in the Google spreadsheet into machine-actionable RDF format, following SKOS standards. The sheet2rdf framework internally evaluates the quality of the SKOS concept scheme using qSKOS \cite{ref-W3C-qSKOS} and deploys the RDF-formatted ontology to OntoStack. 
    
    sheet2rdf is a combination of software used to create a configurable automated process allocated as a GitHub Action workflow in \cite{ref-FAIRsheet2rdf}. The workflow can be manually triggered to update changes in the server.
    
    \item [OntoStack:] The OntoStack framework \cite{ref-OntoStack}  is a set of tools combined for the purpose of handling vocabularies and their RDF properties, such as those yielded by sheet2rdf, and render them in human- and machine-readable formats, and facilitates the deployment and visualisation of the Wind Lidar Ontology.
    
    The Wind Lidar Ontology is an instance of OntoStack and its online visualisation is hosted by the Wind and Energy Systems Department, at the Technical University of Denmark (DTU).
    The wrapper OntoStack gathers functionalities from several tools enumerated here below \cite{ref-github-sheet2rdf}:
    \begin{itemize}
        \item Jena Fuseki \cite{ref-JenaFuseki}: a graph database based on SPARQL, a standard query language and protocol for linked open data on the web \cite{ref-SPARQL} that allows users to store and handle data in RDF format.
        \item Skosmos \cite{ref-Skosmos}: a web-based tool providing services for accessing controlled vocabularies.
        \item Tr\ae fik \cite{ref-Traefik}: an open-source edge router responsible for proper serving of URL requests. 
    \end{itemize}
\end{description}

%\marginpar{[AC 20.12.22] I suggest splitting results and discussion into separate sections.}

\section{Results}
\label{sec:Results}
The details of the Wind Lidar Ontology development procedure are outlined, and the structure and intended uses of the GitHub Wind Lidar Ontology Tool are illustrated.

\subsection{Wind Lidar Ontology}
Structurally inspired by the OpenLidar Architecture, the Wind Lidar Ontology provides a dedicated list of concepts and definitions for supporting development of modular tools and processes for wind lidars. The selection and definition of the concepts included in the Wind Lidar Ontology are the results of the joint effort and consensus of wind lidar experts, combining the information that describes the data and its associated metadata in a way that meets the needs and requirements of both academia and industry.

The development of the Wind Lidar Ontology to its current state consisted of three main phases: 
\begin{enumerate}[]

    \item Defining the main structure of the ontology. This involved establishing the main nodes and relationships among them. First, second and third Wind Lidar Ontology hierarchical levels are shown in Figure \ref{fig:tax}. 

    \item Parent-child relationships among the concepts were fully defined, as well as possible transitive relationships between tuples.

    \item
    Lidar concepts were assigned according to authors' expertise, and the lidar concept definitions and terminology were duly specified. These were subjected to the scrutiny of all members of the ontology group until a consensus was reached. Native speakers proficient in each language provided the translation from the original English version.

\end{enumerate}

Hosting for the online deployment is provided by the Technical University of Denmark (DTU) as an OntoStack instance in the form of a look-up table \cite{TCP32_GitHub_repo}.
The Wind Lidar Ontology homepage includes high-level information about the ontology, as well as direct links to the lidar concepts themselves.

\begin{table}[H] 
\caption{Information contained in each ontology concept.
\label{Ontology_var}}
{\def\arraystretch{2}\tabcolsep=11pt
\begin{tabularx}{\textwidth}{lX}
\toprule
\toprule
\textbf{Variable}&\textbf{Description}		\\
\midrule
\midrule
\textbf{Preferred term}	& 	Name of the variable\\
\hline
\textbf{Definition}	& Consensus definition --- translated to several languages --- briefly describing the ontology concept and its role in the lidar field\\
\hline
\textbf{Broader/Narrower concept}	& If any, hierarchical relation and link to other broader/narrower terms within the ontology\\
\hline
\textbf{Alternative label}	& Other possible names for the term\\
\hline
\textbf{Editorial note}	&  Additional information about the concept \\
\hline
\textbf{In other languages}	& Preferred term in other languages\\
\hline
\textbf{URI}	& A Uniform Resource Identifier as a unique identifier of the physical resource\\
\hline
\textbf{Download this concept} & Machine-readable version of the ontology concept. Downloadable formats: RDF/XML, Turtle and JSON-LD\\
\bottomrule
\bottomrule
\end{tabularx}
}
\end{table}

The Wind Lidar Ontology and its contents may be explored both alphabetically and hierarchically.
The entries within the online version of the ontology contain the fields included in Table \ref{Ontology_var}. $Alternative~label$ contains information about the alternative names associated to a particular lidar term. Each ontology concept has been unambiguously defined and labelled. The labelling has been aligned with the domain of knowledge (for instance, the $Velocity~azimuth~Display$ concept has been labelled as VAD, as it is commonly referred to within the lidar community) and the resulting alternative label can serve either for querying information within the framework or to be downloaded and integrated into digital workflows or codes. 

Queried information is available for download from the DTU server in three different serialisation formats, namely RDF/HTML, Turtle and Json, depending on user preferences.
The following subsection presents a python-based in-house method for downloading and coupling ontology terms to external digital workflows. 

\subsection{The GitHub Wind Lidar Ontology Tool}
\label{GitTool}
The wind lidar concepts and terminology were established according to the process outlined in the Methodology section,
in which opinions were garnered in order to reach a consensus on the content of the Wind Lidar Ontology.
At that stage, a tool was developed to enable the application of metadata to data sets and
allow standardisation across various application domains in wind energy.
The GitHub Wind Lidar Ontology Tool is based on tools developed by the FAIR Data Collective discussed in sub-section \ref{subsec:implementation}, and is openly available at the IEA Wind TCP Task 52 GitHub repository \cite{IEA_Wind_TCP_Task_52_GitHub_repository}.

There are two general approaches to extracting the content of the Wind Lidar Ontology for use within external software applications. 
\begin{enumerate}[label=(\roman*)]

    \item Download individual specific keywords from the webserver-based Wind Lidar Ontology database in the available file formats: RDF; Turtle; JSON.

    \item Directly access the existing Wind Lidar Ontology database file from the GitHub repository. The database may be accessed and filtered through a programming tool, such as Python, for specific keywords.

\end{enumerate}
An example is presented to illustrate the use of the tool.
The general workflow for option (i) is shown in Figure \ref{Ontology_workflow}, where the data flows from the Ontology server (step 1) to the User application (step 5).
\begin{figure}[h!]
    \centering
    \includegraphics[width=0.8\textwidth]{Figures/Ontology_flow.drawio.pdf}
    \caption{General workflow for Ontology concepts}
    \label{Ontology_workflow}
\end{figure}
\newpage
The data is fetched from the Wind Lidar Ontology files in a specific language (e.g., English) according to the workflow in Figure \ref{Ontology_workflow}, which is divided into the following five elements:
\begin{enumerate}
    \item Search the Wind Lidar Ontology database for a keyword (e.g. VAD).
    \item Download the concept as RDF, Turtle or JSON file format.
    \item Extract the desired information from the ontology using the GitHub Wind Lidar Ontology Tool on GitHub (User input: language, labels, keyword).
    \item Store the lidar concept Output and its fields of interest as a python dictionary.
    \item Apply the output (metadata parameters e.g. definition, preferred Label, uri, etc.) to the software application (e.g., lidar simulator or SQL Database).
\end{enumerate}

The \textit{Ontology Viewer} is a DTU hosted webserver where the concept definitions are stored. The code block \textit{Download an Ontology concept} refers to a user action of searching a keyword, selecting the definition and downloading this definition from the Wind Lidar Ontology Viewer. This process is detailed in the code block shown in Figure \ref{fig:download_concept}. 

\begin{figure}[htpb!]
    \centering
    \includegraphics[width=\textwidth]{Figures/download_concept.pdf}
    \caption{Download an Ontology concept}
    \label{fig:download_concept}
\end{figure}

The code block \textit{GitHub Wind Lidar Ontology Tool} represents the python tool developed to extract data from the downloaded definition and making it available for use. The code block \textit{Backend Functions} refers to the python routines that enable the user to fetch definitions.

\begin{figure}[htpb!]
    \centering
    \includegraphics[width=\textwidth]{Figures/extract_keys.pdf}
    \caption{Extracting the keys from a downloaded concept}
    \label{fig:extract_keys}
\end{figure}

Finally, the block \textit{Output} refers to the output from the fetched data that describes the definition of the Ontology's variable. It is a Python dictionary object, which includes uri, type, alternative and preferred labels, definitions and editorial notes, as shown in Figure \ref{fig:extract_keys}.

As an example of a Wind Lidar Ontology application, the flowchart in Figure \ref{fig:workflow_GHT} shows how a local YAML (Yet Another Markup Language is a human-friendly, machine-readable data serialisation language) file, used as an input file for a lidar simulator, is updated with the information extracted from the ontology. By selecting the concept and the fields to be edited, the fetched information is incorporated into the local file. The updating process avoids overwriting, preserving any information contained in the local YAML file.

\begin{figure}[htpb!]
    \centering
    \includegraphics[width=\textwidth]{Figures/flowchart_GH_Tool.png}
    \caption{GitHub Wind Lidar Ontology Tool flowchart. The information extracted from the ontology is used to update a local YAML file, which in turn is used as input for a lidar simulator.  }
    \label{fig:workflow_GHT}
\end{figure}

\newpage
\section{Discussion}
\label{sec:Discussion}
This article has given an overview of the development of the Wind Lidar Ontology, which has been developed with the aim of establishing an industry-wide consensus on wind lidar concepts and terminology. 

The Wind Lidar Ontology was developed by garnering opinions from several parties within the wind energy and lidar industry and academic research groups on the relevant subject matter, and subsequent incorporation into an ontology framework. 
An initial ontology structure was proposed and, subsequently, through a series of meetings, a consensus was reached on what should constitute the final concept definitions and terminology.
A diverse range of viewpoints has ensured that a wide range of concepts and definitions has been captured within the ontology. 

The Wind Lidar Ontology contains more than 200 terms and concepts with definitions, and alternative names where appropriate. In addition, each term definition is given in four languages. 
It is designed to maximize accessibility by utilizing well-established standard data structures such as RDF and the GitHub software platform. The use of RDF in particular makes it an efficient framework for modeling a controlled vocabulary and delineating properties, relationships, constraints, and axioms among its concepts.

It is anticipated that the lidar community and industry will benefit from the Wind Lidar Ontology.
%\subsection{Benefits and limitations}
Ontologies can be integrated into other knowledge infrastructure dedicated to information sharing and integration, thereby enhancing the exchange of information between remote sensing science and the fields of application. Ontologies facilitate opportunities to support interdisciplinary science through better representation and management of scientific knowledge, and wind energy science being fundamentally interdisciplinary, will undoubtedly benefit from such an innovative approach. The Wind Lidar Ontology can be connected to upper level ontologies where appropriate.

In addition, the Wind Lidar Ontology provides standardisation of terminology within the lidar knowledge domain, encourages sharing of inter-domain knowledge and reuse of formalised expert domain knowledge with the minimum of ambiguity. The existence of ontologies should support good industry practices within the wind energy community. Moreover, the Wind Lidar Ontology is a versatile framework which has the adaptability to incorporate new concepts and developments within the lidar industry.
The concepts and terminology to date have been established based on common usage within the industry over several decades.
The open-source framework will promote sharing of the Wind Lidar Ontology and increase the possibility of collaboration through modification and enhancement of concepts and terminology. 
The Wind Lidar Ontology may be further enhanced by translation of concepts and definitions to additional languages.

Despite these recognised benefits, the possible limitations of the Wind Lidar Ontology must be acknowledged.
The software framework may represent a barrier to those users with less experience, and may prove to be difficult to integrate into existing software structures.
Several ontologies exist within the wind energy industry and there exists the possibility that conflicts may arise in the terminology. Such conflicts would be resolved through joint discussion among relevant parties, and usually through ontology alignment.
Additionally, a technological framework such as an ontology requires ownership and maintenance, which requires ongoing commitment in terms of both funding and dedicated time.

\section{Outlook}
\label{sec:Outlook}
The benefits provided to the lidar community by the Wind Lidar Ontology may be enhanced through user engagement and the consolidation of the resulting outcomes, facilitated through the open-source framework.
It is intended that the Wind Lidar Ontology will be updated upon revision and improvement, the updates being published with appropriate version numbering.
In order to expand its use, the Wind Lidar Ontology is being actively promoted to the wind lidar community via webinars, workshops, and publications.
It is anticipated that the wind lidar ontology will evolve over time, and users may submit issues via a
GitHub url \cite{ref-Extract-Wind-Lidar-Ontology-concepts}.

There exists the opportunity to combine the Wind Lidar Ontology with existing (and forthcoming) wind community ontologies, like the ontology  for Wind Turbines and Plants by IEA Wind TCP Task 37 \cite{Task37_web}, or the Ontology for managing data streams by IEA Wind TCP Task 43 \cite{Task43_web}.

The Wind Lidar Ontology presents a promising opportunity to engage neutrally with academia and the wind industry to establish data standards, concepts and terminology. The IEA TCP Task 52 working group on Digitalisation of lidars focuses on this kind of partnership between  research institutions, academia and industry.

\section{Acknowledgments}
%All participants, collectives and institutions that do not appear in the list of authors.
The authors would like to thank Nikola Vasiljevi\'{c}, Julia Gottschall, Peter Clive, the IEA Wind TCP Task 32 and Task 52 participants and the FAIR Collective for their contributions to this work.
The Technical University of Denmark (DTU) is gratefully acknowledged for hosting the Wind Lidar Ontology online.

%%%%%%%%%%%%%%%%%%%%%%%%%%%%%%%%%%%%%%%%%%
\vspace{6pt} 

%%%%%%%%%%%%%%%%%%%%%%%%%%%%%%%%%%%%%%%%%%
%% optional
%\supplementary{The following supporting information can be downloaded at:  \linksupplementary{s1}, Figure S1: title; Table S1: title; Video S1: title.}

% Only for the journal Methods and Protocols:
% If you wish to submit a video article, please do so with any other supplementary material.
% \supplementary{The following supporting information can be downloaded at: \linksupplementary{s1}, Figure S1: title; Table S1: title; Video S1: title. A supporting video article is available at doi: link.}

%%%%%%%%%%%%%%%%%%%%%%%%%%%%%%%%%%%%%%%%%%
\authorcontributions{
%For research articles with several authors, a short paragraph specifying their individual contributions must be provided. The following statements should be used ``Conceptualization, X.X. and Y.Y.; methodology, X.X.; software, X.X.; validation, X.X., Y.Y. and Z.Z.; formal analysis, X.X.; investigation, X.X.; resources, X.X.; data curation, X.X.; writing---original draft preparation, X.X.; writing---review and editing, X.X.; visualization, X.X.; supervision, X.X.; project administration, X.X.; funding acquisition, Y.Y. All authors have read and agreed to the published version of the manuscript.'', please turn to the  \href{http://img.mdpi.org/data/contributor-role-instruction.pdf}{CRediT taxonomy} for the term explanation. Authorship must be limited to those who have contributed substantially to the work~reported.
Conceptualization, F.C., D.L., A.K., A.G., C.R., and A.C.;
methodology, F.C., D.L., A.K., A.G., C.R., and A.C.;
software, F.C. and A.G.; investigation, F.C., D.L., A.K., A.G., C.R., and A.C.;
writing---original draft preparation, F.C., D.L., A.K., A.G., C.R., and A.C.;
writing---review and editing, F.C., D.L., A.K., A.G., C.R., and A.C.;
All authors have read and agreed to the published version of the manuscript.
}

\funding{This research received no external funding.}

% \funding{Please add: "This research received no external funding" or "This research was funded by NAME OF FUNDER grant number XXX." and  "The APC was funded by XXX". Check carefully that the details given are accurate and use the standard spelling of funding agency names at \url{https://search.crossref.org/funding}, any errors may affect your future funding.}

%\institutionalreview{In this section, you should add the Institutional Review Board Statement and approval number, if relevant to your study. You might choose to exclude this statement if the study did not require ethical approval. Please note that the Editorial Office might ask you for further information. Please add “The study was conducted in accordance with the Declaration of Helsinki, and approved by the Institutional Review Board (or Ethics Committee) of NAME OF INSTITUTE (protocol code XXX and date of approval).” for studies involving humans. OR “The animal study protocol was approved by the Institutional Review Board (or Ethics Committee) of NAME OF INSTITUTE (protocol code XXX and date of approval).” for studies involving animals. OR “Ethical review and approval were waived for this study due to REASON (please provide a detailed justification).” OR “Not applicable” for studies not involving humans or animals.}

%\informedconsent{Any research article describing a study involving humans should contain this statement. Please add ``Informed consent was obtained from all subjects involved in the study.'' OR ``Patient consent was waived due to REASON (please provide a detailed justification).'' OR ``Not applicable'' for studies not involving humans. You might also choose to exclude this statement if the study did not involve humans.

%Written informed consent for publication must be obtained from participating patients who can be identified (including by the patients themselves). Please state ``Written informed consent has been obtained from the patient(s) to publish this paper'' if applicable.}


\dataavailability{The Wind Lidar Ontology viewer is available at:  \href{https://data.windenergy.dtu.dk/ontologies/view/ontolidar/en/}{Wind Lidar Ontology viewer}. The GitHub Wind Lidar Ontology Tool is available on GitHub under: \href{https://github.com/PacoCosta/Extract-lidar-ontology-concepts}{GitHub Wind Lidar Ontology Tool}.} 
% \dataavailability{In this section, please provide details regarding where data supporting reported results can be found, including links to publicly archived datasets analyzed or generated during the study. Please refer to suggested Data Availability Statements in section ``MDPI Research Data Policies'' at \url{https://www.mdpi.com/ethics}. If the study did not report any data, you might add ``Not applicable'' here.} 


\conflictsofinterest{The authors declare no conflict of interest.
%The founders had no role in the design of the study; in the collection, analyses, or interpretation of data; in the writing of the manuscript; or in the decision to publish the~results.
}


% Authors must identify and declare any personal circumstances or interest that may be perceived as inappropriately influencing the representation or interpretation of reported research results. Any role of the funders in the design of the study; in the collection, analyses or interpretation of data; in the writing of the manuscript; or in the decision to publish the results must be declared in this section. If there is no role, please state ``The funders had no role in the design of the study; in the collection, analyses, or interpretation of data; in the writing of the manuscript; or in the decision to publish the~results''.} 

%%%%%%%%%%%%%%%%%%%%%%%%%%%%%%%%%%%%%%%%%%
%% Optional
%\sampleavailability{Samples of the compounds ... are available from the authors.}

%% Only for journal Encyclopedia
%\entrylink{The Link to this entry published on the encyclopedia platform.}

\abbreviations{Abbreviations}{
The following abbreviations are used in this manuscript:\\

\noindent 
\begin{tabular}{@{}ll}
FAIR & Findable, Accessible, Interoperable and Reusable\\
GUI & Graphical User Interface\\
Lidar & Light Detection And Ranging\\
OEM & Original Equipment Manufacturer\\
RDF & Resource Description Framework\\
SKOS & Simple Knowledge Organisation System\\
URI & Uniform Resource Identifier\\
YAML & Yet Another Markup Language
\end{tabular}
}

%%%%%%%%%%%%%%%%%%%%%%%%%%%%%%%%%%%%%%%%%%

%%%%%%%%%%%%%%%%%%%%%%%%%%%%%%%%%%%%%%%%%%
\begin{adjustwidth}{-\extralength}{0cm}
%\printendnotes[custom] % Un-comment to print a list of endnotes

\reftitle{References}

% Please provide either the correct journal abbreviation (e.g. according to the “List of Title Word Abbreviations” http://www.issn.org/services/online-services/access-to-the-ltwa/) or the full name of the journal.
% Citations and References in Supplementary files are permitted provided that they also appear in the reference list here. 

%=====================================
% References, variant A: external bibliography
%=====================================
%\bibliography{your_external_BibTeX_file}
\bibliography{references}

%=====================================
% References, variant B: internal bibliography
%=====================================
% \begin{thebibliography}{999}

% \bibitem[Author1(year)]{ref-Liu}
% Liu, Z.;Barlow, J. F.;Chan, P.;Fung, J. C. H.;Li, Y.;Ren, C.;Mak, H. W. L.;Ng, E., A Review of Progress and Applications of Pulsed Doppler Wind LiDARs. {\em Remote Sensing} {\bf 2019}, {\em 11}, 2522.

% \end{thebibliography}


% If authors have biography, please use the format below
%\section*{Short Biography of Authors}
%\bio
%{\raisebox{-0.35cm}{\includegraphics[width=3.5cm,height=5.3cm,clip,keepaspectratio]{Definitions/author1.pdf}}}
%{\textbf{Firstname Lastname} Biography of first author}
%
%\bio
%{\raisebox{-0.35cm}{\includegraphics[width=3.5cm,height=5.3cm,clip,keepaspectratio]{Definitions/author2.jpg}}}
%{\textbf{Firstname Lastname} Biography of second author}

% For the MDPI journals use author-date citation, please follow the formatting guidelines on http://www.mdpi.com/authors/references
% To cite two works by the same author: \citeauthor{ref-journal-1a} (\citeyear{ref-journal-1a}, \citeyear{ref-journal-1b}). This produces: Whittaker (1967, 1975)
% To cite two works by the same author with specific pages: \citeauthor{ref-journal-3a} (\citeyear{ref-journal-3a}, p. 328; \citeyear{ref-journal-3b}, p.475). This produces: Wong (1999, p. 328; 2000, p. 475)

%%%%%%%%%%%%%%%%%%%%%%%%%%%%%%%%%%%%%%%%%%
%% for journal Sci
%\reviewreports{\\
%Reviewer 1 comments and authors’ response\\
%Reviewer 2 comments and authors’ response\\
%Reviewer 3 comments and authors’ response
%}
%%%%%%%%%%%%%%%%%%%%%%%%%%%%%%%%%%%%%%%%%%
\end{adjustwidth}
\end{document}

