%  LaTeX support: latex@mdpi.com 
%  For support, please attach all files needed for compiling as well as the log file, and specify your operating system, LaTeX version, and LaTeX editor.

% see also Google doc at https://docs.google.com/document/d/1Hitne_x4s-Iivetij_XnJBoShJj95FDxYIDc6PRfBDQ/edit#heading=h.2usqgv7yajxv

%=================================================================
\documentclass[remotesensing,article,submit,pdftex,moreauthors]{Definitions/mdpi} 
% For posting an early version of this manuscript as a preprint, you may use "preprints" as the journal and change "submit" to "accept". The document class line would be, e.g., \documentclass[preprints,article,accept,moreauthors,pdftex]{mdpi}. This is especially recommended for submission to arXiv, where line numbers should be removed before posting. For preprints.org, the editorial staff will make this change immediately prior to posting.

%--------------------
% Class Options:
%--------------------
%----------
% journal
%----------
% Choose between the following MDPI journals:
% acoustics, actuators, addictions, admsci, adolescents, aerospace, agriculture, agriengineering, agronomy, ai, algorithms, allergies, alloys, analytica, animals, antibiotics, antibodies, antioxidants, applbiosci, appliedchem, appliedmath, applmech, applmicrobiol, applnano, applsci, aquacj, architecture, arts, asc, asi, astronomy, atmosphere, atoms, audiolres, automation, axioms, bacteria, batteries, bdcc, behavsci, beverages, biochem, bioengineering, biologics, biology, biomass, biomechanics, biomed, biomedicines, biomedinformatics, biomimetics, biomolecules, biophysica, biosensors, biotech, birds, bloods, blsf, brainsci, breath, buildings, businesses, cancers, carbon, cardiogenetics, catalysts, cells, ceramics, challenges, chemengineering, chemistry, chemosensors, chemproc, children, chips, cimb, civileng, cleantechnol, climate, clinpract, clockssleep, cmd, coasts, coatings, colloids, colorants, commodities, compounds, computation, computers, condensedmatter, conservation, constrmater, cosmetics, covid, crops, cryptography, crystals, csmf, ctn, curroncol, currophthalmol, cyber, dairy, data, dentistry, dermato, dermatopathology, designs, diabetology, diagnostics, dietetics, digital, disabilities, diseases, diversity, dna, drones, dynamics, earth, ebj, ecologies, econometrics, economies, education, ejihpe, electricity, electrochem, electronicmat, electronics, encyclopedia, endocrines, energies, eng, engproc, ent, entomology, entropy, environments, environsciproc, epidemiologia, epigenomes, est, fermentation, fibers, fintech, fire, fishes, fluids, foods, forecasting, forensicsci, forests, foundations, fractalfract, fuels, futureinternet, futureparasites, futurepharmacol, futurephys, futuretransp, galaxies, games, gases, gastroent, gastrointestdisord, gels, genealogy, genes, geographies, geohazards, geomatics, geosciences, geotechnics, geriatrics, hazardousmatters, healthcare, hearts, hemato, heritage, highthroughput, histories, horticulturae, humanities, humans, hydrobiology, hydrogen, hydrology, hygiene, idr, ijerph, ijfs, ijgi, ijms, ijns, ijtm, ijtpp, immuno, informatics, information, infrastructures, inorganics, insects, instruments, inventions, iot, j, jal, jcdd, jcm, jcp, jcs, jdb, jeta, jfb, jfmk, jimaging, jintelligence, jlpea, jmmp, jmp, jmse, jne, jnt, jof, joitmc, jor, journalmedia, jox, jpm, jrfm, jsan, jtaer, jzbg, kidney, kidneydial, knowledge, land, languages, laws, life, liquids, literature, livers, logics, logistics, lubricants, lymphatics, machines, macromol, magnetism, magnetochemistry, make, marinedrugs, materials, materproc, mathematics, mca, measurements, medicina, medicines, medsci, membranes, merits, metabolites, metals, meteorology, methane, metrology, micro, microarrays, microbiolres, micromachines, microorganisms, microplastics, minerals, mining, modelling, molbank, molecules, mps, msf, mti, muscles, nanoenergyadv, nanomanufacturing, nanomaterials, ncrna, network, neuroglia, neurolint, neurosci, nitrogen, notspecified, nri, nursrep, nutraceuticals, nutrients, obesities, oceans, ohbm, onco, oncopathology, optics, oral, organics, organoids, osteology, oxygen, parasites, parasitologia, particles, pathogens, pathophysiology, pediatrrep, pharmaceuticals, pharmaceutics, pharmacoepidemiology, pharmacy, philosophies, photochem, photonics, phycology, physchem, physics, physiologia, plants, plasma, pollutants, polymers, polysaccharides, poultry, powders, preprints, proceedings, processes, prosthesis, proteomes, psf, psych, psychiatryint, psychoactives, publications, quantumrep, quaternary, qubs, radiation, reactions, recycling, regeneration, religions, remotesensing, reports, reprodmed, resources, rheumato, risks, robotics, ruminants, safety, sci, scipharm, seeds, sensors, separations, sexes, signals, sinusitis, skins, smartcities, sna, societies, socsci, software, soilsystems, solar, solids, sports, standards, stats, stresses, surfaces, surgeries, suschem, sustainability, symmetry, synbio, systems, taxonomy, technologies, telecom, test, textiles, thalassrep, thermo, tomography, tourismhosp, toxics, toxins, transplantology, transportation, traumacare, traumas, tropicalmed, universe, urbansci, uro, vaccines, vehicles, venereology, vetsci, vibration, viruses, vision, waste, water, wem, wevj, wind, women, world, youth, zoonoticdis 

%---------
% article
%---------
% The default type of manuscript is "article", but can be replaced by: 
% abstract, addendum, article, book, bookreview, briefreport, casereport, comment, commentary, communication, conferenceproceedings, correction, conferencereport, entry, expressionofconcern, extendedabstract, datadescriptor, editorial, essay, erratum, hypothesis, interestingimage, obituary, opinion, projectreport, reply, retraction, review, perspective, protocol, shortnote, studyprotocol, systematicreview, supfile, technicalnote, viewpoint, guidelines, registeredreport, tutorial
% supfile = supplementary materials

%----------
% submit
%----------
% The class option "submit" will be changed to "accept" by the Editorial Office when the paper is accepted. This will only make changes to the frontpage (e.g., the logo of the journal will get visible), the headings, and the copyright information. Also, line numbering will be removed. Journal info and pagination for accepted papers will also be assigned by the Editorial Office.

%------------------
% moreauthors
%------------------
% If there is only one author the class option oneauthor should be used. Otherwise use the class option moreauthors.

%---------
% pdftex
%---------
% The option pdftex is for use with pdfLaTeX. If eps figures are used, remove the option pdftex and use LaTeX and dvi2pdf.

%=================================================================
% MDPI internal commands
\firstpage{1} 
\makeatletter 
\setcounter{page}{\@firstpage} 
\makeatother
\pubvolume{1}
\issuenum{1}
\articlenumber{0}
\pubyear{2022}
\copyrightyear{2022}
%\externaleditor{Academic Editor: Firstname Lastname}
\datereceived{} 
%\daterevised{} % Only for the journal Acoustics
\dateaccepted{} 
\datepublished{} 
%\datecorrected{} % Corrected papers include a "Corrected: XXX" date in the original paper.
%\dateretracted{} % Corrected papers include a "Retracted: XXX" date in the original paper.
\hreflink{https://doi.org/} % If needed use \linebreak
%\doinum{}
%------------------------------------------------------------------
% The following line should be uncommented if the LaTeX file is uploaded to arXiv.org
%\pdfoutput=1

%=================================================================
% Add packages and commands here. The following packages are loaded in our class file: fontenc, inputenc, calc, indentfirst, fancyhdr, graphicx, epstopdf, lastpage, ifthen, lineno, float, amsmath, setspace, enumitem, mathpazo, booktabs, titlesec, etoolbox, tabto, xcolor, soul, multirow, microtype, tikz, totcount, changepage, attrib, upgreek, cleveref, amsthm, hyphenat, natbib, hyperref, footmisc, url, geometry, newfloat, caption

%=================================================================
%% Please use the following mathematics environments: Theorem, Lemma, Corollary, Proposition, Characterization, Property, Problem, Example, ExamplesandDefinitions, Hypothesis, Remark, Definition, Notation, Assumption
%% For proofs, please use the proof environment (the amsthm package is loaded by the MDPI class).

%=================================================================
% Full title of the paper (Capitalized)
\Title{On the Development of an Open-Source Wind Lidar Ontology}

% MDPI internal command: Title for citation in the left column
\TitleCitation{On the Development of an Open-Source Wind Lidar Ontology}

% Author Orchid ID: enter ID or remove command
\newcommand{\orcidauthorA}{0000-0000-0000-000X} % Add \orcidA{} behind the author's name
%\newcommand{\orcidauthorB}{0000-0000-0000-000X} % Add \orcidB{} behind the author's name

% Authors, for the paper (add full first names)
\Author{Firstname Lastname $^{1,\dagger,\ddagger}$\orcidA{}, Firstname Lastname $^{2,\ddagger}$ and Firstname Lastname $^{2,}$*}

%\longauthorlist{yes}

% MDPI internal command: Authors, for metadata in PDF
\AuthorNames{Firstname Lastname, Firstname Lastname and Firstname Lastname}

% MDPI internal command: Authors, for citation in the left column
\AuthorCitation{Lastname, F.; Lastname, F.; Lastname, F.}
% If this is a Chicago style journal: Lastname, Firstname, Firstname Lastname, and Firstname Lastname.

% Affiliations / Addresses (Add [1] after \address if there is only one affiliation.)
\address{%
$^{1}$ \quad Affiliation 1; e-mail@e-mail.com\\
$^{2}$ \quad Affiliation 2; e-mail@e-mail.com}

% Contact information of the corresponding author
\corres{Correspondence: e-mail@e-mail.com; Tel.: (optional; include country code; if there are multiple corresponding authors, add author initials) +xx-xxxx-xxx-xxxx (F.L.)}

% Current address and/or shared authorship
\firstnote{Current address: Affiliation 3.} 
\secondnote{These authors contributed equally to this work.}
% The commands \thirdnote{} till \eighthnote{} are available for further notes

%\simplesumm{} % Simple summary

%\conference{} % An extended version of a conference paper

% Abstract (Do not insert blank lines, i.e. \\) 
\abstract{
This article reports on an open-source ontology which has been developed with the aim of establishing an industry-wide consensus on wind lidar concepts and terminology.
The ontology serves reference and educational purposes.
The article provides an overview of the creation process, the outcomes of the project and the proposed uses of the ontology.
Examples applications are given.
Issues and challenges with writing the ontology are discussed.
}

% Keywords
\keyword{Wind Energy; Lidar; Wind Velocity Measurement; Ontology; Open-Source} 

% The fields PACS, MSC, and JEL may be left empty or commented out if not applicable
%\PACS{J0101}
%\MSC{}
%\JEL{}

%%%%%%%%%%%%%%%%%%%%%%%%%%%%%%%%%%%%%%%%%%
% Only for the journal Diversity
%\LSID{\url{http://}}

%%%%%%%%%%%%%%%%%%%%%%%%%%%%%%%%%%%%%%%%%%
% Only for the journal Applied Sciences
%\featuredapplication{Authors are encouraged to provide a concise description of the specific application or a potential application of the work. This section is not mandatory.}
%%%%%%%%%%%%%%%%%%%%%%%%%%%%%%%%%%%%%%%%%%

%%%%%%%%%%%%%%%%%%%%%%%%%%%%%%%%%%%%%%%%%%
% Only for the journal Data
%\dataset{DOI number or link to the deposited data set if the data set is published separately. If the data set shall be published as a supplement to this paper, this field will be filled by the journal editors. In this case, please submit the data set as a supplement.}
%\datasetlicense{License under which the data set is made available (CC0, CC-BY, CC-BY-SA, CC-BY-NC, etc.)}

%%%%%%%%%%%%%%%%%%%%%%%%%%%%%%%%%%%%%%%%%%
% Only for the journal Toxins
%\keycontribution{The breakthroughs or highlights of the manuscript. Authors can write one or two sentences to describe the most important part of the paper.}

%%%%%%%%%%%%%%%%%%%%%%%%%%%%%%%%%%%%%%%%%%
% Only for the journal Encyclopedia
%\encyclopediadef{For entry manuscripts only: please provide a brief overview of the entry title instead of an abstract.}

%%%%%%%%%%%%%%%%%%%%%%%%%%%%%%%%%%%%%%%%%%
% Only for the journal Advances in Respiratory Medicine
%\addhighlights{yes}
%\renewcommand{\addhighlights}{%

%\noindent This is an obligatory section in “Advances in Respiratory Medicine”, whose goal is to increase the discoverability and readability of the article via search engines and other scholars. Highlights should not be a copy of the abstract, but a simple text allowing the reader to quickly and simplified find out what the article is about and what can be cited from it. Each of these parts should be devoted up to 2~bullet points.\vspace{3pt}\\
%\textbf{What are the main findings?}
% \begin{itemize}[labelsep=2.5mm,topsep=-3pt]
% \item First bullet.
% \item Second bullet.
% \end{itemize}\vspace{3pt}
%\textbf{What is the implication of the main finding?}
% \begin{itemize}[labelsep=2.5mm,topsep=-3pt]
% \item First bullet.
% \item Second bullet.
% \end{itemize}
%}

%%%%%%%%%%%%%%%%%%%%%%%%%%%%%%%%%%%%%%%%%%
\begin{document}

%%%%%%%%%%%%%%%%%%%%%%%%%%%%%%%%%%%%%%%%%%
%\setcounter{section}{-1} %% Remove this when starting to work on the template.
%\section{How to Use this Template}

%The template details the sections that can be used in a manuscript. Note that the order and names of article sections may differ from the requirements of the journal (e.g., the positioning of the Materials and Methods section). Please check the instructions on the authors' page of the journal to verify the correct order and names. For any questions, please contact the editorial office of the journal or support@mdpi.com. For LaTeX-related questions please contact latex@mdpi.com.%\endnote{This is an endnote.} % To use endnotes, please un-comment \printendnotes below (before References). Only journal Laws uses \footnote.

% The order of the section titles is different for some journals. Please refer to the "Instructions for Authors” on the journal homepage.

\section{Introduction}
An ontology defines a common vocabulary within a domain, providing a means to share information in the domain, and includes machine-interpretable definitions of fundamental concepts and the relations among them.
Many disciplines now develop standardised ontologies that domain experts can use to share and annotate information in their fields \cite{ref-Noy}.
An ontology serves several purposes: sharing the common understanding of the structure of information; making the domain assumptions explicit; and, enabling reuse and analysis of the domain knowledge.

Many aspects of wind energy resource assessment and inflow characterization require accurate measurement of wind speed, wind direction and turbulence. In recent years, measurement of wind speeds with lasers, based upon the lidar (Light Detection And Ranging) system, has provided a modern alternative to the meteorological mast. Wind lidar makes use of the principle of optical Doppler shift between the reference radiation and radiation backscattered by aerosol particles to measure radial wind velocities at distances up to several kilometers \cite{ref-Liu}.
Within the field of wind energy, there has been a general move towards adoption of lidar as a cost effective, self-contained and adaptable solution to wind resource measurement.
The use of wind lidars is now widely accepted within the wind energy sector and there has been an associated increase in the technological development of lidar devices and in use cases, and incorporation within a number of IEC standards \cite{ref-IEC61400-12-1, ref-IEC61400-50-3}.

The interest in wind lidar has led to it becoming a research field in its own right, and as a result there has been a multitude of technical advancements and a corresponding plethora of lidar specific concepts and terminology. This situation has given rise to the need for a system to ensure consitency and comprehension within the field.
Further, lidar manufacturers use a variety of different concept definitions, conventions and data formats.
As a result, there exist many difficulties and challenges in communication and lidar knowledge transfer, which may be the case in particular for early stage researchers and engineers.

To tackle the problems, academia and industry has been working for several years and made some progress, but further developments and standardization are still needed. 
A global initiative project e-WindLidar focused on development of wind lidar community based tools and data standards, and publish a lidar data form to make them FAIR, i.e. Findable, Accessible, Interoperable and Reusable \cite{ref-Vasiljevi}. 
The IEA Task 32 (currently the Task 52) bringing together researchers and industry stakeholders to collaborate on the standardization, research and development, and knowledge exchange of wind lidar\cite{ref-Clifton-Schlipf}.
The Openlidar project coordinated by the Stuttgart Wind Energy developed a architecture of an open-source remote wind sensing lidar that can be used for teaching, research, and product development \cite{ref-SWE}.
Nevertheless, only limited efforts are seen in the past years and it become quite necessary to speed up the process since lidar community is expanding quickly and more early researchers and engineers join.

In recognition of this, an open-source ontology has been developed with the aim of establishing an industry-wide consensus on wind lidar concepts and terminology.
Begin as a subtask of the IEA task 32, some fundamental works of lidar ontology was accomplished \cite{ref-Clifton-Costa}. Afterwards, the task continues on the IEA task 52 and a small group consist of academia and industry engineers keep working on it. After a year of consistent bi-weekly meetings and extra communications and effors, a latest version of lidar ontology is published online \cite{ref-OntoWeb}
Aim to create guidelines for a common conceptual architecture for lidar design, modelling and analysis, the lidar ontology includes fundamental introduction of lidar hardware and software, lidar installation and measurement, data generation and process, and examples and use cases, etc. The purpose is to reduce ambiguity, to make wind lidar easier to understand and use. With the ontology, researchers and engineers can collaborate and reuse of the domain knowledge, avoiding possible pitfalls. 

This article provides an introduction to the wind lidar ontology and gives an overview of its development, and summarises the outcomes.
How to use the lidar ontology. The lidar ontology has been developed as an open-source code repository, thereby  being freely available for possible modification and redistribution by users.
As such, users can download definitions and contribute to the lidar ontology programme.
Consequently, the lidar ontology will be subject to ongoing development with release version numbers.

The paper is structured as follows. Section \ref{sec:Methodology} provides an overview of the ontology methodology. The results are presented and discussed in Section \ref{sec:Results_and_discussion} and the conclusions and outlook are presented in Section \ref{sec:Conclusion}.


\section{Methodology}
\label{sec:Methodology}
 Controlled vocabularies (CVs) allow a clear and neat representation of data sets, and we can use them to store, structure, maintain and share information. We may understand a CV as a collection of words for a specific domain of knowledge, ordered in a certain way. You may consider also a concept, as each term of this collection. Resource Description Framework (RDF) is a general-purpose language[citation W3C RDF Recommendations], and is commonly used to model a CV and define relationships, properties, constrains and axioms among its concepts. Therefore, RDF allows us to digitise, express and handle knowledge organisation systems in a machine-readable format. Simple Knowledge Organisation System (SKOS) offers standards to create data models for metadata, and jointly with RDF allows us to serve machines with standardise data organisation models.

\subsection{Taxonomy}

A taxonomy is a type of CV in which concepts, including their descriptive metadata,  are linked in a hierarchical order or sorted into categories.

The first step in building a lidar taxonomy is to create a list of terms, or concepts, related to the wind lidar knowledge domain. To do so, authors followed the "Expert elicitation" (or "Top-down") approach (see Figure \ref{tax}) , in which experts in the relevant topics of a sector are gathered for collaborating in defining the hierarchy of terms (top topics) defining the main branches of the hierarchical  tree starting  from the broad term  i.e.  “wind lidar” down to the narrower terms[citation: IRPWind: Taxonomy and meta data for wind energy R&D]. A structured categorisation of this terms according to certain premises results in a wind lidar taxonomy. 

\begin{figure}[h]
    \centering
    \includegraphics[width=12cm]{MindMap.PNG}
    
    \caption{First, second (Design) and third lidar taxonomy hierarchical levels.}
    \label{tax}
\end{figure}


\subsection{Ontology }
An ontology is another type of CV with a higher level of detailed information and stronger constrains among concepts than in a taxonomy. If we endow the lidar concepts with extended information and enable the whole data structure to be machine-readable, we uplift the taxonomy to an ontology. The wrappers sheet2rdf [citation] and ontostack[citation] have been used to be able  to orchestrate the framework necessary to deploy the lidar ontology.


\subsection{Implementation}
The ontology is eventually deployed by means of a combination of a different tools, brought together using GitHub actions for an automatic compilation and deployment. The workflow is as follows: 
\begin{description}
    
    \item [Input data sheet:] The input data sheet consists of a Google Sheets excel sheet stored in Google Drive. The repository  \textit{IEA-Wind-Task-32/wind-lidar-ontology} hosting the source files for the wind lidar ontology [citation] contains a template of this excel sheet as an asset. A google sheet format for the input data ease the implementation of new data and its maintenance, compare to other more complex and less intuitive machine-readable formats like RDF. The google sheet has the required format to be ultimately read and handled by sheet2rdf to be converted to an RDF vocabulary.
    The authors have filled in the excel sheet with the lidar concepts and the related descriptive metadata considered, so far, the most relevant for the wind lidar community. 
    
    \item [sheet2rdf:] sheet2rdf builds on excel2rdf [citation] workflow and it is a GitHub repository hosting the main functions that enables the conversion of the information contained in the google sheet into machine-actionable RDF format vocabulary, following Simple Knowledge Organisation System (SKOS) standards. 
    
    The framework internally evaluates the quality of the SKOS concept scheme using qSKOS [citation]. As a last step, sheet2rdf framework deploys the RDF-formatted ontology to ontostack and ensures the update of the deployed ontology when the excel sheet changes.
    
    \item [ontostack:] ontostack is a set of tools combined for the purpose of handling vocabularies and their RDF properties, as such yielded by sheet2rdf, and put them to the service of human and machines. It allows the deployment and visualisation of the wind lidar ontology. In the present case the developed wind lidar ontology is an instance of ontostack and its online visualisation is hosted by the Wind and Energy Systems Department, at the Technical University  of Denmark (DTU).
    The ontostack workflow contains some tools like [citation: https://github.com/nikokaoja/sheet2rdf]:
    \begin{itemize}
        \item Jena Fuseki: a graph database
        \item SKOSMOS: kosmos is a web-based tool providing services for accessing controlled vocabularies [citation: https://github.com/NatLibFi/Skosmos]
        \item Tr\ae fik: an edge router responsible for proper serving of URL requests
    \end{itemize}
\end{description}

\marginpar{[AC 20.12.22] I suggest splitting results and discussion into separate sections.}
\section{Results and discussion}
\label{sec:Results_and_discussion}
Results and discussion

All figures and tables should be cited in the main text as Figure~\ref{fig1}, Table~\ref{tab1}, etc.

\begin{figure}[H]
\includegraphics[width=10.5 cm]{Definitions/logo-mdpi}
\caption{This is a figure. Schemes follow the same formatting. If there are multiple panels, they should be listed as: (\textbf{a}) Description of what is contained in the first panel. (\textbf{b}) Description of what is contained in the second panel. Figures should be placed in the main text near to the first time they are cited. A caption on a single line should be centered.\label{fig1}}
\end{figure}   
\unskip

\begin{table}[H] 
\caption{This is a table caption. Tables should be placed in the main text near to the first time they are~cited.\label{tab1}}
\newcolumntype{C}{>{\centering\arraybackslash}X}
\begin{tabularx}{\textwidth}{CCC}
\toprule
\textbf{Title 1}	& \textbf{Title 2}	& \textbf{Title 3}\\
\midrule
Entry 1		& Data			& Data\\
Entry 2		& Data			& Data\\
\bottomrule
\end{tabularx}
\end{table}
\unskip

\subsection{Lidar ontology}
Lidar ontology

\subsubsection{Overview}
Overview

\subsection{Examples}
Examples

\subsubsection{Transfer data, tools, and knowledgeOpen library}
Example

\subsubsection{Standardize input files}
Example

% ----------
% Opportunities and Challenges
% ----------
% [AC 20.12.22] introduced new section to replace / capture material from conclusions
\section{Opportunities and Challenges}

\section{Conclusion}
\label{sec:Conclusion}
Conclusion

\subsection{Issues and Challenges}
Issues and challenges
\marginpar{[AC 20.12.22] Conclusions should not introduce new material. Might be better to have a new section called "Discussion" with this }

\subsection{Outlook}
Outlook


%%%%%%%%%%%%%%%%%%%%%%%%%%%%%%%%%%%%%%%%%%
\vspace{6pt} 

%%%%%%%%%%%%%%%%%%%%%%%%%%%%%%%%%%%%%%%%%%
%% optional
%\supplementary{The following supporting information can be downloaded at:  \linksupplementary{s1}, Figure S1: title; Table S1: title; Video S1: title.}

% Only for the journal Methods and Protocols:
% If you wish to submit a video article, please do so with any other supplementary material.
% \supplementary{The following supporting information can be downloaded at: \linksupplementary{s1}, Figure S1: title; Table S1: title; Video S1: title. A supporting video article is available at doi: link.}

%%%%%%%%%%%%%%%%%%%%%%%%%%%%%%%%%%%%%%%%%%
\authorcontributions{For research articles with several authors, a short paragraph specifying their individual contributions must be provided. The following statements should be used ``Conceptualization, X.X. and Y.Y.; methodology, X.X.; software, X.X.; validation, X.X., Y.Y. and Z.Z.; formal analysis, X.X.; investigation, X.X.; resources, X.X.; data curation, X.X.; writing---original draft preparation, X.X.; writing---review and editing, X.X.; visualization, X.X.; supervision, X.X.; project administration, X.X.; funding acquisition, Y.Y. All authors have read and agreed to the published version of the manuscript.'', please turn to the  \href{http://img.mdpi.org/data/contributor-role-instruction.pdf}{CRediT taxonomy} for the term explanation. Authorship must be limited to those who have contributed substantially to the work~reported.}

\funding{Please add: ``This research received no external funding'' or ``This research was funded by NAME OF FUNDER grant number XXX.'' and  and ``The APC was funded by XXX''. Check carefully that the details given are accurate and use the standard spelling of funding agency names at \url{https://search.crossref.org/funding}, any errors may affect your future funding.}

\institutionalreview{In this section, you should add the Institutional Review Board Statement and approval number, if relevant to your study. You might choose to exclude this statement if the study did not require ethical approval. Please note that the Editorial Office might ask you for further information. Please add “The study was conducted in accordance with the Declaration of Helsinki, and approved by the Institutional Review Board (or Ethics Committee) of NAME OF INSTITUTE (protocol code XXX and date of approval).” for studies involving humans. OR “The animal study protocol was approved by the Institutional Review Board (or Ethics Committee) of NAME OF INSTITUTE (protocol code XXX and date of approval).” for studies involving animals. OR “Ethical review and approval were waived for this study due to REASON (please provide a detailed justification).” OR “Not applicable” for studies not involving humans or animals.}

\informedconsent{Any research article describing a study involving humans should contain this statement. Please add ``Informed consent was obtained from all subjects involved in the study.'' OR ``Patient consent was waived due to REASON (please provide a detailed justification).'' OR ``Not applicable'' for studies not involving humans. You might also choose to exclude this statement if the study did not involve humans.

Written informed consent for publication must be obtained from participating patients who can be identified (including by the patients themselves). Please state ``Written informed consent has been obtained from the patient(s) to publish this paper'' if applicable.}

\dataavailability{In this section, please provide details regarding where data supporting reported results can be found, including links to publicly archived datasets analyzed or generated during the study. Please refer to suggested Data Availability Statements in section ``MDPI Research Data Policies'' at \url{https://www.mdpi.com/ethics}. If the study did not report any data, you might add ``Not applicable'' here.} 

\acknowledgments{In this section you can acknowledge any support given which is not covered by the author contribution or funding sections. This may include administrative and technical support, or donations in kind (e.g., materials used for experiments).}

\conflictsofinterest{Declare conflicts of interest or state ``The authors declare no conflict of interest.'' Authors must identify and declare any personal circumstances or interest that may be perceived as inappropriately influencing the representation or interpretation of reported research results. Any role of the funders in the design of the study; in the collection, analyses or interpretation of data; in the writing of the manuscript; or in the decision to publish the results must be declared in this section. If there is no role, please state ``The funders had no role in the design of the study; in the collection, analyses, or interpretation of data; in the writing of the manuscript; or in the decision to publish the~results''.} 

%%%%%%%%%%%%%%%%%%%%%%%%%%%%%%%%%%%%%%%%%%
%% Optional
\sampleavailability{Samples of the compounds ... are available from the authors.}

%% Only for journal Encyclopedia
%\entrylink{The Link to this entry published on the encyclopedia platform.}

\abbreviations{Abbreviations}{
The following abbreviations are used in this manuscript:\\

\noindent 
\begin{tabular}{@{}ll}
MDPI & Multidisciplinary Digital Publishing Institute\\
DOAJ & Directory of open access journals\\
TLA & Three letter acronym\\
LD & Linear dichroism
\end{tabular}
}

%%%%%%%%%%%%%%%%%%%%%%%%%%%%%%%%%%%%%%%%%%
%% Optional
\appendixtitles{no} % Leave argument "no" if all appendix headings stay EMPTY (then no dot is printed after "Appendix A"). If the appendix sections contain a heading then change the argument to "yes".
\appendixstart
\appendix
\section[\appendixname~\thesection]{}
\subsection[\appendixname~\thesubsection]{}
The appendix is an optional section that can contain details and data supplemental to the main text---for example, explanations of experimental details that would disrupt the flow of the main text but nonetheless remain crucial to understanding and reproducing the research shown; figures of replicates for experiments of which representative data are shown in the main text can be added here if brief, or as Supplementary Data. Mathematical proofs of results not central to the paper can be added as an appendix.

\begin{table}[H] 
\caption{This is a table caption.\label{tab5}}
\newcolumntype{C}{>{\centering\arraybackslash}X}
\begin{tabularx}{\textwidth}{CCC}
\toprule
\textbf{Title 1}	& \textbf{Title 2}	& \textbf{Title 3}\\
\midrule
Entry 1		& Data			& Data\\
Entry 2		& Data			& Data\\
\bottomrule
\end{tabularx}
\end{table}

\section[\appendixname~\thesection]{}
All appendix sections must be cited in the main text. In the appendices, Figures, Tables, etc. should be labeled, starting with ``A''---e.g., Figure A1, Figure A2, etc.

%%%%%%%%%%%%%%%%%%%%%%%%%%%%%%%%%%%%%%%%%%
\begin{adjustwidth}{-\extralength}{0cm}
%\printendnotes[custom] % Un-comment to print a list of endnotes

\reftitle{References}

% Please provide either the correct journal abbreviation (e.g. according to the “List of Title Word Abbreviations” http://www.issn.org/services/online-services/access-to-the-ltwa/) or the full name of the journal.
% Citations and References in Supplementary files are permitted provided that they also appear in the reference list here. 

%=====================================
% References, variant A: external bibliography
%=====================================
%\bibliography{your_external_BibTeX_file}

%=====================================
% References, variant B: internal bibliography
%=====================================
\begin{thebibliography}{999}

\bibitem[Author1(year)]{ref-Noy}
Noy, N.F.; McGuinness, D.L. Ontology Development 101: A Guide to Creating Your First Ontology. Available online: https://protege.stanford.edu/publications/ontology\_development/ontology101.pdf (accessed on 14 October 2022).

\bibitem[Author1(year)]{ref-Liu}
Liu, Z.;Barlow, J. F.;Chan, P.;Fung, J. C. H.;Li, Y.;Ren, C.;Mak, H. W. L.;Ng, E., A Review of Progress and Applications of Pulsed Doppler Wind LiDARs. {\em Remote Sensing} {\bf 2019}, {\em 11}, 2522.

\bibitem[Author1(year)]{ref-IEC61400-12-1}
IEC 61400-12-1:2017, Wind power generation systems - 
Part 12-1: Power performance measurement of electricity producing wind turbines. {\em BSI Standards Publication} {\bf 2017}.

\bibitem[Author1(year)]{ref-IEC61400-50-3}
IEC 61400-50-3, Wind energy generation systems – Part 50-3: Use of nacelle-mounted lidars for wind measurements. {\em BSI Standards Publication} {\bf 2021}.

\bibitem[Author1(year)]{ref-Vasiljevi}
Vasiljevic, N., Klaas, T., Pauscher, L., Gomes, D. F., Abreuand, R., Bardal, L. M., Klaas, T., Pauscher, L., Lopes, C., Fer-, D., Abreu, R., & Morten, L. e-WindLidar : making wind lidar data FAIR. {\em https://doi.org/10.5281/zenodo.2478051} {\bf 2018}.

\bibitem[Author1(year)]{ref-Clifton-Schlipf}
Clifton, Andrew, Schlipf, David, Vasiljevic, Nikola, Gottschall, Julia, Clive, Peter, Wüerth, Ines, Wagner, Rozenn, & Nygaard, Nicolai. (2020). IEA Wind Task 32: Collaborative R&D Roadmap (2020 September 15). Zenodo. https://doi.org/10.5281/zenodo.4030701

\bibitem[Author1(year)]{ref-SWE}
Stuttgart Wind Energy. https://www.openlidar.net/the-project/ (access on 05/02/2023) {\bf 2021}.

\bibitem[Author1(year)]{ref-Clifton-Costa}
Clifton, Andrew, Costa, Francisco, & Vasiljevic, Nikola. (2021, June 30). A wind lidar ontology to help information exchange.  {\em Zenodo} {\bf 2021}. https://doi.org/10.5281/zenodo.5046720

\bibitem[Author1(year)]{ref-OntoWeb} https://data.windenergy.dtu.dk/ontologies/view/ontolidar/en/


\end{thebibliography}

% If authors have biography, please use the format below
%\section*{Short Biography of Authors}
%\bio
%{\raisebox{-0.35cm}{\includegraphics[width=3.5cm,height=5.3cm,clip,keepaspectratio]{Definitions/author1.pdf}}}
%{\textbf{Firstname Lastname} Biography of first author}
%
%\bio
%{\raisebox{-0.35cm}{\includegraphics[width=3.5cm,height=5.3cm,clip,keepaspectratio]{Definitions/author2.jpg}}}
%{\textbf{Firstname Lastname} Biography of second author}

% For the MDPI journals use author-date citation, please follow the formatting guidelines on http://www.mdpi.com/authors/references
% To cite two works by the same author: \citeauthor{ref-journal-1a} (\citeyear{ref-journal-1a}, \citeyear{ref-journal-1b}). This produces: Whittaker (1967, 1975)
% To cite two works by the same author with specific pages: \citeauthor{ref-journal-3a} (\citeyear{ref-journal-3a}, p. 328; \citeyear{ref-journal-3b}, p.475). This produces: Wong (1999, p. 328; 2000, p. 475)

%%%%%%%%%%%%%%%%%%%%%%%%%%%%%%%%%%%%%%%%%%
%% for journal Sci
%\reviewreports{\\
%Reviewer 1 comments and authors’ response\\
%Reviewer 2 comments and authors’ response\\
%Reviewer 3 comments and authors’ response
%}
%%%%%%%%%%%%%%%%%%%%%%%%%%%%%%%%%%%%%%%%%%
\end{adjustwidth}
\end{document}

